% slides.tex
\documentclass[20pt]{beamer}
\usepackage{listings}
\usepackage[utf8]{inputenc}
\usepackage{cancel}
\usepackage{color}
\usepackage{graphicx}
\usepackage{pdfpages}
\usepackage{amsmath}

\usetheme{default}
\usecolortheme{dove}
\useoutertheme{default}

% Slightly smaller title
\setbeamerfont{frametitle}{size=\large}
\setbeamerfont{verb}{size=\small}

% Font
\renewcommand{\ttdefault}{pcr}

% lst settings
\lstset{
    basicstyle=\ttfamily\small,
    gobble=4,
    keywordstyle=\ttfamily\bfseries,
    language=Haskell,
    showstringspaces=false
}

\newcommand{\vspaced}{
    \vspace{5mm}
}

\newcommand{\imageframe}[1]{
    {
        \usebackgroundtemplate{
            \includegraphics[height=\paperheight,width=\paperwidth]{#1}}
        \setbeamertemplate{navigation symbols}{}
        \begin{frame}[plain]
        \end{frame}
    }
}

\newcommand{\code}[1]{
    \texttt{\small{#1}}
}

\begin{document}

\title{The Ludwig DSL}
\subtitle{Haskell Exchange 2015}
\author{Jasper Van der Jeugt}
\date{October 8, 2015}

\begin{frame}[plain]
    \titlepage
\end{frame}

% Introduction
% ------------

\begin{frame}{Hello!}
    My name is Jasper \\
    I write Haskell \\
    Currently for Luminal \\
    \vspaced
    \texttt{@jaspervdj} \\
    \texttt{jaspervdj.be}
\end{frame}

\imageframe{images/fugue.jpg}

\begin{frame}{Fugue}
    Fugue builds, optimizes, and enforces cloud infrastructure -- continuously
    and automatically. Declare your cloud and let Fugue automate the rest.
\end{frame}

\begin{frame}[plain]
    \center{How do you ``declare a cloud''?}
\end{frame}

\begin{frame}[fragile,plain]
    \begin{lstlisting}
    my-app:
      services: [
        internet,
        web-frontend,
        database-backend
      ]
    \end{lstlisting}
\end{frame}

\begin{frame}[plain]
    \center{Why not YAML?}
\end{frame}

\begin{frame}[fragile]{Why not YAML?}
    \textbf{Abstraction}
    \vspaced
    \begin{lstlisting}[keywords = {}]
    tasks:
      - command: echo {{ item }}
        with_items: [ 0, 2, 4 ]
        when: item > 2
    \end{lstlisting}
\end{frame}

\begin{frame}[fragile]{Why not YAML?}
    \textbf{Where do you store defaults?} \\
    \vspaced
    In the parser? \\
    In another configuration file? \\
    And also in the documentation? \\
\end{frame}

\begin{frame}[fragile]{Why not YAML?}
    \textbf{How do you document?} \\
    \vspaced
    Documentation is hard to write \\
    \dots and hard to maintain \\
\end{frame}

\begin{frame}[fragile]{Why not YAML?}
    \textbf{How do you provide error messages?} \\
    \vspaced
    \begin{lstlisting}[keywords = {}]
    $ ./foo bar.yaml
    wrong type for field 'delay';
    integer
    \end{lstlisting}
\end{frame}

\begin{frame}{So what is Ludwig?}
    YAML-like syntax \\
    Statically typed \\
    Algebraic datatypes \\
    Extensible records \\
\end{frame}

\begin{frame}{So what is Ludwig?}
    Functions \\
    Pattern matching \\
    Modules \\
\end{frame}

% Here we start to talk about typing
% ----------------------------------

\imageframe{images/overview-typechecker.pdf}

\begin{frame}[fragile]{Typing}
    \begin{lstlisting}
    vanilla:
      name:  "Vanilla milkshake"
      price: 5

    strawberry: vanilla {
      name: "Strawberry milkshake"
    }
    \end{lstlisting}
\end{frame}

\begin{frame}[fragile]{Typing}
    \begin{lstlisting}
    data CoreExpr
      = ...
      | LitE Literal
      | ...

    infer (LitE (StringLit _)) =
      return stringType
    infer (LitE (IntLit _)) =
      return intType
    \end{lstlisting}
\end{frame}

\begin{frame}[fragile]{Typing}
    \begin{lstlisting}
    vanilla:
      name:  "Vanilla milkshake"
      price: 5
    \end{lstlisting}
\end{frame}

\imageframe{images/overview-desugar.pdf}

\begin{frame}[fragile,plain]
    \begin{lstlisting}
    data Expr
      = ...
      | StructE [(Key, Expr)]
      | ...

    data CoreExpr
      = ...
      | EmptyE
      | SetE Key Expr Expr
      | ...
    \end{lstlisting}
\end{frame}

\begin{frame}[fragile,plain]
    \begin{lstlisting}
    desugar expr = case expr of
      ...
      StructE ((k, e) : xs)) ->
        SetE k e (desugar xs)
      StructE []             ->
        EmptyE
      ...
    \end{lstlisting}
\end{frame}

\begin{frame}[fragile,plain]
    \begin{lstlisting}
    vanilla:
      set("name",
        "Vanilla milkshake",
      set("price", 5,
      empty))
    \end{lstlisting}
\end{frame}

\imageframe{images/overview-typechecker.pdf}

\begin{frame}[fragile,plain]
    \[
    \begin{array}{rclcrcl}
    \tau & ::= & \text{C}               & ~~ & \sigma & ::= & \forall
                                                    \bar{\alpha}. \tau    \\
         &  |  & \alpha                 & ~~ &        &     &             \\
         &  |  & \tau ~ \tau            & ~~ &        &     &             \\
         &  |  & \tau \rightarrow \tau  & ~~ & \rho   & ::= & \emptyset   \\
         &  |  & \{ \rho \}             & ~~ &        &  |  & \code{k:}\tau
                                                                | \rho    \\
    \end{array}
    \]
\end{frame}

\begin{frame}[fragile,plain]
    \begin{lstlisting}
    infer EmptyE =
      return $ RowT EmptyR

    infer (SetE k e r) =
      alpha <- freshTyVar
      t     <- infer e
      rt    <- infer r
      unify alpha rt
      return $ RowT $
        ExtendR k e alpha
    \end{lstlisting}
\end{frame}

\begin{frame}[fragile,plain]
    \begin{lstlisting}
    vanilla:
      set("name",
        "Vanilla milkshake",
      set("price", 5,
      empty))
    \end{lstlisting}
\end{frame}

\begin{frame}[fragile,plain]
    \[
    \code{infer(vanilla)} = \alpha
    \]
    \[
    \begin{array}{rclll}
    \alpha  & = & \{ \code{name:}  & \code{string} & | ~ \beta  \} \\
    \beta   & = & \{ \code{price:} & \code{int}    & | ~ \gamma \} \\
    \gamma  & = & \emptyset                                        \\
    \end{array}
    \]
\end{frame}

\begin{frame}[fragile,plain]
    \[
    \code{infer(vanilla)}
    \]
    \[
    \begin{array}{lll}
    ~ & = & \{ \code{name:} \code{string} ~ | ~
               \code{price:} \code{int} ~ | ~ \emptyset \} \\
    ~ & = & \{ \code{name:} \code{string}, ~ \code{price:} \code{int} \} \\
    \end{array}
    \]
\end{frame}

\begin{frame}[fragile,plain]
    \begin{lstlisting}
    vanilla:
      name:  "Vanilla milkshake"
      price: 5

    strawberry: vanilla {
      name: "Strawberry milkshake"
    }
    \end{lstlisting}
\end{frame}

\begin{frame}[fragile,plain]
    \begin{lstlisting}
    strawberry:
      set("name",
        "Strawberry milkshake",
      vanilla)
    \end{lstlisting}
\end{frame}


\begin{frame}[fragile,plain]
    \[
    \code{infer(strawberry)} = \alpha
    \]
    \[
    \begin{array}{rcl}
    \alpha & = & \{ \code{name:} \code{string} ~ | ~ \beta  \} \\
    \beta  & = & \{ \code{name:} \code{string}, ~ \code{price:} \code{int} \} \\
    \end{array}
    \]
\end{frame}

\begin{frame}[fragile,plain]
    \[
    \code{infer(strawberry)}
    \]
    \[
    \begin{array}{llll}
    ~ & = & \{ & \code{name:} \code{string}, \\
      &   &    & \code{name:} \code{string}, ~ \code{price:} \code{int} \} \\
      & = & \{ & \code{name:} \code{string}, ~ \code{price:} \code{int} \} \\
    \end{array}
    \]
\end{frame}

\begin{frame}{Other approaches}
    Scoped labels \\
    Predicates to indicate label absence \\
\end{frame}

\begin{frame}[fragile]{Use case: partial defaults}
    \begin{lstlisting}[keywords = {}]
    default-milkshake:
      price: 5
    \end{lstlisting}
\end{frame}

\begin{frame}[fragile]{Use case: partial defaults}
    \begin{lstlisting}[keywords = {}]
    vanilla: default-milkshake {
      name: "Vanilla milkshake"
    }

    chocolate: default-milkshake {
      name:  "Chocolate milkshake",
      price: 6
    }
    \end{lstlisting}
\end{frame}

% Here we start to talk about a module system
% -------------------------------------------

\begin{frame}[fragile]{Separate type definitions}
    \textbf{In \texttt{milklib.lw}:}
    \begin{lstlisting}[keywords = {}]
    alias milkshake:
      name:  string
      price: int

    default-milkshake:
      price: 5
    \end{lstlisting}
\end{frame}

\begin{frame}[fragile]{Separate type definitions}
    \textbf{In \texttt{menu.lw}:}
    \begin{lstlisting}[keywords = {}]
    vanilla: default-milkshake {
      name: "Vanilla milkshake"
    }

    chocolate: default-milkshake {
      name:  "Chocolate milkshake",
      price: 6
    }
    \end{lstlisting}
\end{frame}

\imageframe{images/overview-scopechecker.pdf}

\begin{frame}[fragile]{Separate type definitions}
    Implementing a module system:
    \vspaced
    \begin{enumerate}
    \item Check overall module sanity
    \item Resolve names everywhere
    \end{enumerate}
\end{frame}

\begin{frame}[fragile]{Check overall module sanity}
    The one base function which does everything!**
\end{frame}

\begin{frame}[fragile]{Check overall module sanity}
    The one base function which does everything!** \\
    \vspaced
    * Does not actually do everything \\
    * In \texttt{containers}, not \texttt{base} \\
\end{frame}

\begin{frame}[fragile,plain]
    \begin{lstlisting}
    stronglyConnComp
      :: Ord key
      => [(node, key, [key])]
      -> [SCC node]

    data SCC n
      = AcyclicSCC n
      | CyclicSCC [n]
    \end{lstlisting}
\end{frame}

\begin{frame}[fragile]{Resolve names everywhere}
    \begin{lstlisting}
    type Scope =
      Trie String NameInfo

    data NameInfo = NameInfo
      { niModuleName :: String
      , niNamespace  :: Namespace
      , ...
      }
    \end{lstlisting}
\end{frame}

\begin{frame}[fragile]{Resolve names everywhere}
    \begin{lstlisting}
    moduleToScope
      :: Module -> Scope
    \end{lstlisting}
    \vspaced
    \begin{lstlisting}[keywords = {}]
    [ milkshake: NameInfo
        "milklib" TypeName TopLevel
    , default-milkshake: NameInfo
        "milklib" VarName TopLevel
    ]
    \end{lstlisting}
\end{frame}

\begin{frame}[fragile]{Quantification}
    \begin{lstlisting}
    requires ml: milklib
    \end{lstlisting}
\end{frame}

\imageframe{images/trie-01.pdf}

\imageframe{images/trie-02.pdf}

\begin{frame}[fragile]{Quantification}
    \begin{lstlisting}[keywords = {}]
    quantify
      :: String -> Scope -> Scope

    [ ml.milkshake: NameInfo
        "milklib" TypeName TopLevel
    , ml.default-milkshake: NameInfo
        "milklib" VarName TopLevel
    ]
    \end{lstlisting}
\end{frame}

\begin{frame}[fragile]{Validation}
    \begin{lstlisting}[keywords = {}]
    union
      :: Scope
      -> Scope
      -> Either
            [ScopeError]
            Scope
    \end{lstlisting}
\end{frame}

\begin{frame}[fragile]{Validation}
    \begin{lstlisting}
    data Validation e a
      = Ok   a
      | Fail e

    instance Functor
        (Validation e) where
      fmap f (Ok x)   = Ok (f x)
      fmap _ (Fail e) = Fail e
    \end{lstlisting}
\end{frame}

\begin{frame}[fragile,plain]
    \begin{lstlisting}
    instance Monoid e =>
        Applicative (Validation e)
        where
      pure = Ok

      Ok   f  <*> Ok   x  = Ok (f x)
      Ok   _  <*> Fail e  = Fail e
      Fail e  <*> Ok   _  = Fail e
      Fail e1 <*> Fail e2 =
        Fail (e1 <> e2)
    \end{lstlisting}
\end{frame}

\begin{frame}[fragile]{Validation}
    \begin{lstlisting}[keywords = {}]
    union
      :: Scope
      -> Scope
      -> Validation
            [ScopeError]
            Scope
    \end{lstlisting}
\end{frame}

\begin{frame}[fragile,plain]
    \begin{lstlisting}
    data Expr n
      = VarE n
      | LamE n (Expr n)
      | ...

    scopeCheck
      :: Scope
      -> Expr Name
      -> Validation
           [ScopeError]
           (Expr FullName)
    \end{lstlisting}
\end{frame}

\begin{frame}[fragile,plain]
    \begin{lstlisting}
    scopeCheck s (VarE n) =
      lookupVar s n

    scopeCheck s (LamE n e) =
      LamE
        <$> pure n'
        <*> scopeCheck s' e
     where
      (n', s') = bindLocalName n s
    \end{lstlisting}
\end{frame}

\begin{frame}{Scopechecking}
    \textbf{Disadvantages:} \\
    Extra phase, extra code \\
    \vspaced
    \textbf{Advantages:} \\
    List of things we don't care about:
    duplicate names, shadowing warnings,
    (type) variable not found\dots
\end{frame}

\begin{frame}[fragile,plain]
    Bonus: documentation generation
\end{frame}

\imageframe{images/overview-docs.pdf}

\begin{frame}[fragile,plain]
    \begin{lstlisting}[keywords = {}]
    # A yummy drink
    alias milkshake:
      # Name of the yummy drink
      name:  string
      # Price of the yummy drink
      price: int
    \end{lstlisting}
\end{frame}

\begin{frame}[fragile,plain]
    Problem: comments are removed by tokenization \\
    \vspaced
    \begin{lstlisting}
    parseAlias = do
      Token.reserved "alias"
      ...
    \end{lstlisting}
\end{frame}

\begin{frame}[fragile,plain]
    Solution: store comments as we parse them \\
    \vspaced
    \begin{lstlisting}
    whiteSpace
      :: MonadState CommentBlock s
      => ParsecT s u m ()
    \end{lstlisting}
\end{frame}

\begin{frame}[fragile,plain]
    Solution: store comments as we parse them \\
    \vspaced
    \begin{lstlisting}
    whiteSpace = do
      skipMany space
      optional $ do
        comment <- commentBlock
        putCommentBlock comment
        whiteSpace
    \end{lstlisting}
\end{frame}

\begin{frame}[fragile,plain]
    \begin{lstlisting}
    parseAlias = do
      Token.reserved "alias"
      comment <- getCommentBlock
      ...

    data Alias = Alias
        { aName     :: Name
        , aMetadata :: Metadata
        , ...
        }
    \end{lstlisting}
\end{frame}

% Questions?
% ----------

\begin{frame}[plain]
    \center{\huge{Questions?}}
\end{frame}

\end{document}
