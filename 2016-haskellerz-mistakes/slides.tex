% slides.tex
\documentclass[20pt]{beamer}
\usepackage{listings}
\usepackage[utf8]{inputenc}
\usepackage{cancel}
\usepackage{color}
\usepackage{graphicx}
\usepackage{pdfpages}
\usepackage{amsmath}
\usepackage{xcolor}

\usetheme{default}
\usecolortheme{dove}
\useoutertheme{default}

% Remove navigation symbols
\setbeamertemplate{navigation symbols}{}

% Slightly smaller title
\setbeamerfont{frametitle}{size=\large}
\setbeamerfont{verb}{size=\small}

% Font
\renewcommand{\ttdefault}{pcr}

% lst settings
\lstset{
    basicstyle=\ttfamily\small,
    gobble=4,
    keywordstyle=\ttfamily\bfseries,
    showstringspaces=false
}

\newcommand{\vspaced}{
    \vspace{5mm}
}

\newcommand{\imageframe}[1]{
    {
        \usebackgroundtemplate{
            \includegraphics[height=\paperheight,width=\paperwidth]{#1}}
        \setbeamertemplate{navigation symbols}{}
        \begin{frame}[plain]
        \end{frame}
    }
}

\newcommand{\chapterslide}[1]{
    {
        \begin{frame}[plain]
        \begin{center}
        \large{#1}
        \end{center}
        \end{frame}
    }
}

\newcommand{\code}[1]{
    \texttt{\small{#1}}
}

\definecolor{asparagus}{rgb}{0.53, 0.66, 0.42}
\definecolor{bittersweet}{rgb}{1.0, 0.44, 0.37}

\begin{document}

\title{Haskell:\\
Mistakes I've made}
\subtitle{HaskellerZ}
\author{Jasper Van der Jeugt}
\date{March 31, 2016}

\begin{frame}[plain]
    \titlepage
\end{frame}

\begin{frame}{About me}
    \begin{itemize}
    \item I really like Haskell
    \item Contributed to and authored open source projects
    \item Did some consulting for various companies
    \item Worked at Better, currently at \code{fugue.co}
    \end{itemize}
\end{frame}

\begin{frame}{What this talk is about}
    \emph{Beginner} Haskell ``Mistakes'' that can be avoided, signs of Haskell
    code smell.
\end{frame}

\begin{frame}[fragile]{What this talk is not about}
    We only talk about code that is accepted by the compiler.
    \vspaced
    \begin{lstlisting}
    fib :: Int -> Int
    fib n =
      | n <= 1    = 1
      | otherwise =
          fib (n - 1) +
          fib (n - 2)
    \end{lstlisting}
\end{frame}

% Pick your battles
% -----------------

\chapterslide{Intro: pick your battles}

\begin{frame}{Pick your battles}
    It's often impossible to have perfectly clean code \emph{everywhere} in
    large projects.
\end{frame}

\begin{frame}{Pick your battles}
    \begin{itemize}
    \item Aesthetically pleasing code?
    \item Just need to get the job done?
    \item Sacrifice clean code on the performance altar?
    \item Document top-level functions?
    \end{itemize}
\end{frame}

\begin{frame}{Pick your battles}
    What works for me: focus on \emph{modules} as units.
\end{frame}

\begin{frame}{Pick your battles: modules}
    The \emph{exposed interface} of a module should be clean. \\
    \vspaced
    \emph{\small{But inside you can \code{unsafePerformIO} all you want.}}
\end{frame}

\begin{frame}{Pick your battles: modules}
    The module should have some documentation about what it is for and how one
    should use it. \\
    \vspaced
    \emph{\small{But you don't necessarily need to write Haddock for all
    functions if it's clear what they do from the name.}}
\end{frame}

\begin{frame}{Pick your battles: modules}
    There should be some tests that verify that the module works correctly. \\
    \vspaced
    \emph{\small{But you don't need to test all the internals.}}
\end{frame}

% Mistake: not using `-Wall`
% --------------------------

\chapterslide{Mistake: not using \code{-Wall}}

\begin{frame}{Mistake: not using \code{-Wall}}
    \emph{Always} compile using \code{-Wall}. \\
    Use \code{-Werror} for production/tests.
\end{frame}

\begin{frame}{Low-risk \code{-Wall}}
    \small{
    \texttt{-fwarn-unused-imports},
    \texttt{-fwarn-dodgy-exports},
    \texttt{-fwarn-dodgy-imports},
    \texttt{-fwarn-monomorphism-restriction},
    \texttt{-fwarn-implicit-prelude},
    \texttt{-fwarn-missing-local-sigs},
    \texttt{-fwarn-missing-exported-sigs},
    \texttt{-fwarn-missing-import-lists},
    \texttt{-fwarn-identities}
    }
\end{frame}

\begin{frame}{Medium-risk \code{-Wall}}
    \small{
    \texttt{-fwarn-unused-binds},
    \texttt{-fwarn-unused-matches},
    \texttt{-fwarn-auto-orphans}
    }
\end{frame}

\begin{frame}{High-risk \code{-Wall}}
    \small{
    \texttt{-fwarn-incomplete-patterns}
    \texttt{-fwarn-incomplete-uni-patterns}
    \texttt{-fwarn-incomplete-record-updates}
    }
\end{frame}

% Mistake: wildcard pattern matches
% ---------------------------------

\chapterslide{Mistake: wildcard pattern matches}

\begin{frame}[fragile]{Mistake: wildcard pattern matches}
    \begin{lstlisting}
    data Murderer
      = Innocent
      | Murderer

    canBeReleased
      :: IsMurderer -> Bool
    canBeReleased = \case
      Innocent -> True
      Murderer -> False
    \end{lstlisting}
\end{frame}

\begin{frame}[fragile]{Mistake: wildcard pattern matches}
    \begin{lstlisting}
    fireSquad :: IsMurderer -> Bool
    fireSquad = \case
      Innocent -> False
      _        -> True
    \end{lstlisting}
\end{frame}

\begin{frame}[fragile]{Mistake: wildcard pattern matches}
    \begin{lstlisting}
    data IsMurderer
      = Innocent
      | Murderer
      | Like99PercentSure
    \end{lstlisting}
\end{frame}

\begin{frame}[fragile]{Mistake: wildcard pattern matches}
    Note: this also appears \\
    in other forms!
    \vspaced
    \begin{lstlisting}
    fireSquad :: IsMurderer -> Bool
    fireSquad x = x /= Innocent
    \end{lstlisting}
\end{frame}

\begin{frame}{Mistake: wildcard pattern matches}
    It is often best to explicitly match on every constructor, unless you can be
    sure no further constructors will be added (\code{Maybe}, \code{Either}...).
\end{frame}

% Mistake: you think you understand exceptions
% --------------------------------------------

\chapterslide{Mistake: you think you understand exceptions}

\begin{frame}{Mistake: exceptions}
    Most common mistake: you think you understand how exceptions work in
    Haskell. \\
    \vspaced
    \small{Alternatively: you falsely assume you will always pay attention
    to how they work.}
\end{frame}

\begin{frame}[fragile]{Laziness + exceptions = hard}
    \begin{lstlisting}
    errOrOk <- try $
      return [1, 2, error "kaputt"]
    case errOrOk of
      Left err -> print err
      Right x  -> print x
    \end{lstlisting}
\end{frame}

\begin{frame}[fragile]{Laziness + exceptions = hard}
    \begin{lstlisting}
    errOrOk <- try $
      return (throw MyException)
    case errOrOk of
      Left err -> print err
      Right x  -> print x
    \end{lstlisting}
\end{frame}

\begin{frame}[fragile]{Async exceptions: very hard}
    Does this function close the socket?
    \begin{lstlisting}
    foo :: (Socket -> IO a) -> IO a
    foo f = do
      s <- openSocket
      r <- try $ f s
      closeSocket s
      ...
    \end{lstlisting}
\end{frame}

\begin{frame}[fragile]{Async exceptions: very hard}
    How about:
    \begin{lstlisting}
    foo :: (Socket -> IO a) -> IO a
    foo f = mask $ \restore -> do
      s <- openSocket
      r <- try $ restore $ f s
      closeSocket s
      ...
    \end{lstlisting}
\end{frame}

\begin{frame}{Debugging async exceptions}
    \begin{enumerate}
    \normalsize{\item Easy, this function shouldn't throw exceptions.}
    \small{\item The exception comes from another place? Let's use
        \texttt{mask}?}
    \footnotesize{\item Maybe we can set \texttt{unsafeUnmask} to allow an
        interrupt?}
    \footnotesize{\item ...}
    \tiny{\item Nothing works and programming is an eternal cycle of pain and
        darkness.}
    \tiny{\item Go back to 1}
    \end{enumerate}
\end{frame}

\begin{frame}{Exceptions: avoiding the madness}
    Keep your code pure where possible.
\end{frame}

\begin{frame}{Exceptions: avoiding the madness}
    Try to avoid throwing exceptions from pure code. If you have an exception
    lurking somewhere deep within a thunk, bad things will usually happen.
\end{frame}

\begin{frame}{Exceptions: avoiding the madness}
    Use more predictable monad stacks:
    \begin{itemize}
    \item \code{IO (Either e a)}
    \item \code{Control.Monad.Except}
    \end{itemize}
\end{frame}

\begin{frame}{Exceptions: avoiding the madness}
    Use existing solutions such as \code{resource-pool}, \code{resourcet}, and
    existing patterns like \code{bracket}.
\end{frame}

% Avoiding GHC Extensions
% -----------------------

\chapterslide{Mistake: avoiding GHC extensions}

\begin{frame}{Mistake: avoiding GHC extensions}
    Before I wrote Haskell: C/C++ background. Lots of pain trying to get things
    compiling on MSVC++.
\end{frame}

\begin{frame}{Mistake: avoiding GHC extensions}
    \begin{enumerate}
    \item There is only GHC
    \item Extensions widely used in ``standard'' packages
    \item It's Haskell so refactoring is easy
    \end{enumerate}
\end{frame}

\begin{frame}{Mistake: avoiding GHC extensions}
    Average of around 3 extensions per module. Commonly:
    \code{OverloadedStrings},
    \code{TemplateHaskell},
    \code{DeriveDataTypeable},
    \code{ScopedTypeVariables},
    \code{GeneralizedNewtypeDeriving},
    \code{BangPatterns}
\end{frame}

\begin{frame}{Mistake: avoiding GHC extensions}
    Extensions that improve readability are always a good idea! \\
    \vspaced
    \code{LambdaCase},
    \code{ViewPatterns},
    \code{PatternGuards},
    \code{MultiWayIf},
    \code{TupleSections},
    \code{BinaryLiterals}
\end{frame}

\begin{frame}{Mistake: avoiding GHC extensions}
    Extensions that write code for you are even better! \\
    \vspaced
    \code{DeriveFunctor}, \\
    \code{DeriveFoldable},
    \code{DeriveTraversable},
    \code{RecordWildCards},
    \code{DeriveGeneric}
\end{frame}

\begin{frame}{Mistake: avoiding GHC extensions}
    Further reading: \\
    \vspaced
    \emph{Oliver Charles: \\
    24 days of GHC extensions}
\end{frame}

% Mistake: not testing enough
% ---------------------------

\chapterslide{Mistake: not testing enough}

\begin{frame}{Mistake: not testing enough}
    \begin{center}
    \emph{If it compiles, ship it.}
    \end{center}
\end{frame}

\begin{frame}[fragile]{Mistake: not testing enough}
    QuickCheck/SmallCheck \\
    \vspaced
    \begin{lstlisting}
    prop_revRev xs =
      reverse (reverse xs) == xs
    \end{lstlisting}
\end{frame}

\begin{frame}{Mistake: not testing enough}
    Lots of tutorials available on QuickCheck/SmallCheck, but many people only
    use this to test a \emph{pure core} of their application.
\end{frame}

\begin{frame}{Mistake: not testing enough}
    Test your entire application! \\
    \vspaced
    \begin{itemize}
    \item \code{hspec-snap}
    \item \code{hspec-webdrivers}
    \item \code{tasty-golden}
    \item \ldots
    \end{itemize}
\end{frame}

\begin{frame}{Mistake: not testing enough}
    Test your entire application! \\
    \vspaced
    \begin{itemize}
    \item Roll your own framework
    \item Bash
    \item Python
    \item JavaScript
    \item \ldots
    \end{itemize}
\end{frame}

% Mistake: wrong string types
% ---------------------------

\chapterslide{Mistake: string types}

\begin{frame}{Mistake: string types}
    \begin{enumerate}
    \item Write some code
    \item It's pretty slow
    \item Find a stackoverflow thread from 2008 that tells me to use
        \code{ByteString}
    \item Z\boxed{?}rich
    \end{enumerate}
\end{frame}

\begin{frame}{Mistake: string types}
    \begin{center}
    \includegraphics[width=\textwidth]{images/strings.pdf}
    \end{center}
\end{frame}

\begin{frame}{Mistake: string types}
    \begin{center}
    \includegraphics[width=\textwidth]{images/strings-lazy.pdf}
    \end{center}
\end{frame}

% Mistake: not enough pure code
% -----------------------------

\chapterslide{Mistake: not enough pure code}

\begin{frame}[fragile]{Mistake: not enough pure code}
    \begin{lstlisting}
    main :: IO ()
    main = do
      x <- readLn :: IO Int
      y <- readLn :: IO Int
      -- wizzling
      let wizzled =
            x + y * y * 3 + 9
      print wizzled
    \end{lstlisting}
\end{frame}

\begin{frame}[fragile]{Pure code: sometimes it's easy}
    \begin{lstlisting}
    wizzle :: Int -> Int -> Int
    wizzle x y =
      x + y * y * 3 + 9
    \end{lstlisting}
\end{frame}

\begin{frame}[fragile]{Pure code: sometimes it's easy}
    \begin{lstlisting}
    main :: IO ()
    main = do
      x <- readLn :: IO Int
      y <- readLn :: IO Int
      print $ wizzled x y
    \end{lstlisting}
\end{frame}

\begin{frame}{Pure code: sometimes it's easy}
    \begin{enumerate}
    \item Read your input
    \item Do your stuff
    \item Write your output
    \end{enumerate}
\end{frame}

\begin{frame}{Pure code: sometimes it's hard}
    \begin{itemize}
    \item Can't read all input because of memory constraints
    \item Input reading depends on temporary results
    \item Need for threads, continuous output
    \item \ldots
    \end{itemize}
\end{frame}

\begin{frame}[fragile]{Pure code: example}
    \begin{lstlisting}
    data Request = Request
      { rqPath :: String
      , ...
      }

    data Response
      = Response200 String
      | Response404
    \end{lstlisting}
\end{frame}

\begin{frame}[fragile]{Pure code: example}
    \begin{lstlisting}
    handler
      :: Request -> IO Response
    handler = do
      rq <- readRequest
      case rqPath rq of
        "/index" ->
          return $ Response200 "ok"
        _        ->
          return Response404
    \end{lstlisting}
\end{frame}

\begin{frame}{Pure code: example}
    \emph{Startup idea}: People can't be expected to reverse their own strings
    if they are on a low-powered device like an IoT-enabled toaster. We should
    have a cloud service for reversing a string.
\end{frame}

\begin{frame}[fragile]{Pure code: example}
    \begin{lstlisting}
    data Request = Request
      { rqPath :: String
      , rqBody :: String
      , ...
      }

    data Response
      = Response200 String
      | Response404
    \end{lstlisting}
\end{frame}

\begin{frame}[fragile]{Pure code: example}
    \begin{lstlisting}
    handler = do
      rq <- readRequest
      case rqPath rq of
        "/rev"   ->
          return $ Response200 $
            reverse (rqBody rq)
        ...
    \end{lstlisting}
\end{frame}

\begin{frame}[fragile]{Pure code: example}
    \begin{lstlisting}
    data Request = Request
      { rqPath :: String
      , ...
      }

    data Response
      = Response200 String
      | Response404
      | ReadBody
          (String -> Response)
    \end{lstlisting}
\end{frame}

\begin{frame}[fragile]{Pure code: example}
    \begin{lstlisting}
    handler = do
      rq <- readRequest
      case rqPath rq of
        "/rev"   -> return $
          ReadBody $ \body ->
            Response200 $
            reverse body
    \end{lstlisting}
\end{frame}

\begin{frame}[fragile]{Pure code: example}
    \begin{lstlisting}
    run (Response200 str) =
      sendResponse str
    run Response404 =
      error "404"
    run (ReadBody f) = do
      body <- readRequestBody
      run (f body)
    \end{lstlisting}
\end{frame}

\begin{frame}[fragile]{Pure code: free}
    \begin{lstlisting}
    data Free (f :: * -> *) a
      = Free (f (Free f a))
      | Pure a
    \end{lstlisting}
\end{frame}

\begin{frame}[fragile]{Pure code: free}
    \begin{lstlisting}
    data Response
      = Response200 String
      | Response404

    data HandlerF a
      = ReadBodyF (String -> a)
      deriving (Functor)

    type Handler =
        Free HandlerF
    \end{lstlisting}
\end{frame}

\begin{frame}[fragile]{Pure code: free}
    \begin{lstlisting}
    getBody :: Handler String
    getBody = Free $
        ReadBodyF return

    ok200
      :: String
      -> Handler Response
    ok200 str = Pure $
        Response200 str
    \end{lstlisting}
\end{frame}

\begin{frame}[fragile]{Pure code: free}
    \begin{lstlisting}
    handler
      :: Request
      -> Handler Response
    handler rq =
      case rqPath rq of
        "/rev" -> do
          str <- getBody
          ok200 $ reverse str
    \end{lstlisting}
\end{frame}

\begin{frame}[fragile]{Pure code: free}
    \begin{lstlisting}
    run :: Handler Response -> IO ()
    run (Pure (Response200 str)) =
      sendResponse str
    run (Pure (Response404)) =
      error "404"
    run (Free (ReadBodyF f)) = do
      body <- readRequestBody
      run (f body)
    \end{lstlisting}
\end{frame}

% Mistake: overdoing abstraction
% ------------------------------

\chapterslide{Mistake: overdoing abstraction}

\begin{frame}[fragile]{Mistake: overdoing abstraction}
    \begin{lstlisting}[basicstyle=\ttfamily\footnotesize]
    zygoHistoPrepro
      :: (Unfoldable t, Foldable t)
      => (Base t b -> b)
      -> (forall c. Base t c -> Base t c)
      -> (Base t (EnvT b (Stream (Base t)) a)
            -> a)
      -> t -> a
    zygoHistoPrepro f g t =
      -- unless you want a generalized
      -- zygomorphism.
      gprepro (distZygoT f distHisto) g t
    \end{lstlisting}
\end{frame}

\begin{frame}[fragile]{Mistake: overdoing abstraction}
    \begin{lstlisting}
    getProp k =
      case getProperty k m of
        Left  err -> fail err
        Right val ->
          return (show val)
    \end{lstlisting}
\end{frame}

\begin{frame}[fragile]{Mistake: overdoing abstraction}
    Or do you prefer this version? \\
    \vspaced
    \begin{lstlisting}
    getProp =
      either fail (return . show) .
      flip getProperty
    \end{lstlisting}
\end{frame}

% Mistake: premature abstraction
% ------------------------------

\chapterslide{Mistake: Premature abstraction}

\begin{frame}{Mistake: premature abstractions}
    Example: the resource provider in \emph{Hakyll}. \\
    \vspaced
    An abstaction around the file system to list and obtain resources as various
    types. It provides some caching and access to metadata.
\end{frame}

\begin{frame}{Mistake: premature abstractions}
    \begin{itemize}
    \item Zero or one instances
    \item Lots of code working around restrictions
    \item Making assumptions about specific behaviours
    \item Too general
    \item Too abstract
    \end{itemize}
\end{frame}

\begin{frame}[fragile]{Mistake: premature abstractions}
    (Historically incorrect) example: \\
    \vspaced
    \begin{lstlisting}
    class Functor m => Monad m where
      return :: a -> m a
      join   :: m (m a) -> m a

      -- wat
      fail   :: String -> m a
    \end{lstlisting}
\end{frame}

\begin{frame}{Mistake: premature abstractions}
    Abstractions are very hard to get right. \\
    \vspaced
    In general, just writing and repeating the non-abstract code until you
    \emph{really} see the pattern helps.
\end{frame}

\chapterslide{Questions?}

\end{document}
